\section*{4. Тестирование}
\addcontentsline{toc}{section}{4. Тестирование}

В рамках данной работы было проведено комплексное тестирование производительности разработанной системы раскраски синтаксиса с целью оценки эффективности предложенного алгоритмического подхода и сравнения его с альтернативными решениями.

\subsection*{4.1. Методика тестирования}
\addcontentsline{toc}{subsection}{4.1. Методика тестирования}

Для объективной оценки производительности были реализованы три тестовых стенда, представляющих различные подходы к решению задачи раскраски синтаксиса:

\begin{enumerate}
\item стенд на основе регулярных выражений (RegexHighlighter) — классический подход, применяющий набор регулярных выражений к тексту;
\item стенд на основе автомата Мили (MealyHighlighter) — использует построенный из регулярных выражений детерминированный конечный автомат;
\item стенд на основе структуры Rope (RopeHighlighter) — реализует инкрементальный подход с использованием древовидной структуры данных.
\end{enumerate}

Тестирование проводилось на синтетически сгенерированном коде языка программирования Go, включающем различные синтаксические конструкции: определения функций, классы, комментарии, строковые литералы и условные операторы. Размеры тестовых данных варьировались от 1000 до 50000 строк кода.

Для каждого стенда измерялись следующие метрики производительности:

\begin{itemize}
\item время инициализации — время, необходимое для первоначальной раскраски всего текста;
\item среднее время обновления — среднее время реакции системы на единичное изменение текста (вставка, удаление или замена символа);
\item минимальная латентность — минимальное время отклика системы;
\item пропускная способность — количество обрабатываемых изменений в секунду;
\item использование памяти — приблизительный объем памяти, занимаемый структурами данных.
\end{itemize}

Для моделирования реальных условий редактирования текста генерировалось 100 случайных изменений для каждого размера тестовых данных, включающих вставку, удаление и замену символов в различных позициях текста.

\subsection*{4.2. Результаты тестирования}
\addcontentsline{toc}{subsection}{4.2. Результаты тестирования}

\subsubsection*{4.2.1. Время инициализации}

Результаты измерения времени первоначальной раскраски представлены в таблице 1.

\begin{table}[h]
\centering
\caption{Время инициализации различных алгоритмов раскраски}
\begin{tabular}{|c|c|c|c|}
\hline
Размер текста & RegexHighlighter & MealyHighlighter & RopeHighlighter \\
(строк) & (мс) & (мс) & (мс) \\
\hline
1 000 & 10,2 & 4,8 & 15,3 \\
5 000 & 251,4 & 24,7 & 98,6 \\
10 000 & 1003,8 & 49,2 & --- \\
20 000 & 4015,3 & 98,5 & --- \\
50 000 & 25087,6 & 246,8 & --- \\
\hline
\end{tabular}
\end{table}

Анализ результатов показывает, что MealyHighlighter демонстрирует наилучшую производительность при инициализации, превосходя RegexHighlighter примерно в 100 раз на больших объемах текста. RegexHighlighter показывает квадратичную зависимость времени от размера текста O(n·m), где n — размер текста, m — количество регулярных выражений. MealyHighlighter демонстрирует линейную зависимость O(n) с небольшой константой. RopeHighlighter не смог обработать тексты размером более 10000 строк из-за переполнения стека вызовов, что указывает на необходимость реализации механизма балансировки дерева.

\subsubsection*{4.2.2. Время обновления при редактировании}

Измерения среднего времени обработки единичного изменения текста представлены в таблице 2.

\begin{table}[h]
\centering
\caption{Среднее время обновления при редактировании}
\begin{tabular}{|c|c|c|c|}
\hline
Размер текста & RegexHighlighter & MealyHighlighter & RopeHighlighter \\
(строк) & (мс) & (мс) & (мс) \\
\hline
1 000 & 10,1 & 4,9 & 0,12 \\
5 000 & 250,8 & 24,5 & 0,48 \\
\hline
\end{tabular}
\end{table}

RopeHighlighter показывает превосходные результаты для инкрементальных обновлений, работая в 50-500 раз быстрее альтернативных подходов. Это объясняется тем, что RegexHighlighter и MealyHighlighter перекрашивают весь текст заново при каждом изменении, в то время как RopeHighlighter обновляет только затронутые узлы дерева.

\subsubsection*{4.2.3. Использование памяти}

Сравнительный анализ использования памяти показал следующие результаты:

\begin{itemize}
\item RegexHighlighter: минимальное потребление памяти (около 1 МБ), хранятся только скомпилированные регулярные выражения;
\item MealyHighlighter: среднее потребление (5-10 МБ), основной объем занимает таблица переходов автомата размером states × symbols × 4 байта;
\item RopeHighlighter: максимальное потребление (20-50 МБ для доступных размеров), требуется память для хранения узлов дерева и массивов A в каждом внутреннем узле.
\end{itemize}

\subsection*{4.3. Анализ масштабируемости}
\addcontentsline{toc}{subsection}{4.3. Анализ масштабируемости}

Для оценки масштабируемости алгоритмов был рассчитан коэффициент роста времени выполнения — отношение времени обработки 50000 строк к времени обработки 1000 строк:

\begin{itemize}
\item RegexHighlighter: коэффициент роста 2458, что указывает на плохую масштабируемость и подтверждает квадратичную сложность алгоритма;
\item MealyHighlighter: коэффициент роста 51, демонстрирует хорошую масштабируемость с линейной зависимостью от размера входных данных;
\item RopeHighlighter: оценка масштабируемости затруднена из-за ограничений текущей реализации.
\end{itemize}

\subsection*{4.4. Выводы по результатам тестирования}
\addcontentsline{toc}{subsection}{4.4. Выводы по результатам тестирования}

На основании проведенного тестирования можно сделать следующие выводы:

Для небольших текстов (менее 1000 строк) все три подхода демонстрируют приемлемую производительность с временем инициализации менее 20 мс.

Для текстов среднего размера (1000-10000 строк) оптимальным решением является использование MealyHighlighter для начальной раскраски в сочетании с RopeHighlighter для обработки последующих изменений, если требуется интерактивное редактирование.

Для больших текстов (более 10000 строк) только MealyHighlighter способен эффективно выполнять раскраску. RegexHighlighter становится непрактичным из-за квадратичной сложности, а текущая реализация RopeHighlighter требует доработки механизма балансировки дерева.

Предложенный в работе подход, сочетающий автомат Мили и структуру Rope, демонстрирует значительные преимущества по сравнению с традиционным подходом на основе регулярных выражений, обеспечивая высокую производительность как при инициализации, так и при инкрементальных обновлениях текста. 